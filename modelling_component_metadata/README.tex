\documentclass[]{article}
\usepackage{lmodern}
\usepackage{amssymb,amsmath}
\usepackage{ifxetex,ifluatex}
\usepackage{fixltx2e} % provides \textsubscript
\ifnum 0\ifxetex 1\fi\ifluatex 1\fi=0 % if pdftex
  \usepackage[T1]{fontenc}
  \usepackage[utf8]{inputenc}
\else % if luatex or xelatex
  \ifxetex
    \usepackage{mathspec}
  \else
    \usepackage{fontspec}
  \fi
  \defaultfontfeatures{Ligatures=TeX,Scale=MatchLowercase}
\fi
% use upquote if available, for straight quotes in verbatim environments
\IfFileExists{upquote.sty}{\usepackage{upquote}}{}
% use microtype if available
\IfFileExists{microtype.sty}{%
\usepackage{microtype}
\UseMicrotypeSet[protrusion]{basicmath} % disable protrusion for tt fonts
}{}
\usepackage{hyperref}
\hypersetup{unicode=true,
            pdfborder={0 0 0},
            breaklinks=true}
\urlstyle{same}  % don't use monospace font for urls
\IfFileExists{parskip.sty}{%
\usepackage{parskip}
}{% else
\setlength{\parindent}{0pt}
\setlength{\parskip}{6pt plus 2pt minus 1pt}
}
\setlength{\emergencystretch}{3em}  % prevent overfull lines
\providecommand{\tightlist}{%
  \setlength{\itemsep}{0pt}\setlength{\parskip}{0pt}}
\setcounter{secnumdepth}{0}
% Redefines (sub)paragraphs to behave more like sections
\ifx\paragraph\undefined\else
\let\oldparagraph\paragraph
\renewcommand{\paragraph}[1]{\oldparagraph{#1}\mbox{}}
\fi
\ifx\subparagraph\undefined\else
\let\oldsubparagraph\subparagraph
\renewcommand{\subparagraph}[1]{\oldsubparagraph{#1}\mbox{}}
\fi

\date{}

\begin{document}

\section{Modelling Component
Metadata}\label{modelling-component-metadata}

M1: Make your components, profiles and concepts ``as generic as possible
and as specific as needed''

{[}priority: high{]}

Although it is hard to provide precise guidelines for the degree to
which a component or profile should be generally applicable, it is
strongly recommended to always consider the potential of reuse even when
you are modelling for a specific project or domain. The general
principle of being ``as generic as possible and as specific as needed''
is reflected in many of the best practices in this section and
throughout this document. For example, when choosing names for your
elements you have the choice whether to use project specific names or
more context agnostic ones; when deciding whether an element should be
multilingual or not, you have the choice to only consider the metadata
you will be creating but also to increase the usability of your profile
by enabling this option in suitable places regardless. Therefore,
although not very prescriptive in and of itself, we consider this to be
one of the core best practices when it comes to CMDI modelling.

\end{document}
