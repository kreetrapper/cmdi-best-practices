\documentclass[]{article}
\usepackage{lmodern}
\usepackage{amssymb,amsmath}
\usepackage{ifxetex,ifluatex}
\usepackage{fixltx2e} % provides \textsubscript
\ifnum 0\ifxetex 1\fi\ifluatex 1\fi=0 % if pdftex
  \usepackage[T1]{fontenc}
  \usepackage[utf8]{inputenc}
\else % if luatex or xelatex
  \ifxetex
    \usepackage{mathspec}
  \else
    \usepackage{fontspec}
  \fi
  \defaultfontfeatures{Ligatures=TeX,Scale=MatchLowercase}
\fi
% use upquote if available, for straight quotes in verbatim environments
\IfFileExists{upquote.sty}{\usepackage{upquote}}{}
% use microtype if available
\IfFileExists{microtype.sty}{%
\usepackage{microtype}
\UseMicrotypeSet[protrusion]{basicmath} % disable protrusion for tt fonts
}{}
\usepackage{hyperref}
\hypersetup{unicode=true,
            pdfborder={0 0 0},
            breaklinks=true}
\urlstyle{same}  % don't use monospace font for urls
\IfFileExists{parskip.sty}{%
\usepackage{parskip}
}{% else
\setlength{\parindent}{0pt}
\setlength{\parskip}{6pt plus 2pt minus 1pt}
}
\setlength{\emergencystretch}{3em}  % prevent overfull lines
\providecommand{\tightlist}{%
  \setlength{\itemsep}{0pt}\setlength{\parskip}{0pt}}
\setcounter{secnumdepth}{0}
% Redefines (sub)paragraphs to behave more like sections
\ifx\paragraph\undefined\else
\let\oldparagraph\paragraph
\renewcommand{\paragraph}[1]{\oldparagraph{#1}\mbox{}}
\fi
\ifx\subparagraph\undefined\else
\let\oldsubparagraph\subparagraph
\renewcommand{\subparagraph}[1]{\oldsubparagraph{#1}\mbox{}}
\fi

\date{}

\begin{document}

\section{Introduction}\label{introduction}

In 2016, both version 1.2\footnote{See https://www.clarin.eu/cmdi1.2} of
the
\href{https://www.gitbook.com/book/clarin-eric/cmdi-best-practices/edit\#}{Component
Metadata} (CMD) Infrastructure (CMDI) and a first complete technical
specification (CE-2016-0880) of this metadata standard were introduced.
The new version introduced new possibilities, which have been gradually
opened up by the ecosystem of tools and registries in CMDI. One of the
key properties of CMDI is its flexibility, which makes it possible to
create metadata records closely tailored to the requirements of
resources and tools/services. However, design and implementation choices
made at various levels in the CMD lifecycle might influence how well or
easily a CMD record is processed and its associated resources made
available in the CLARIN infrastructure. Knowledge on this has
traditionally been scattered around in various documents, web pages and
even completely hidden from sight in experts' minds. To make this
knowledge explicit, the CMDI and Metadata Curation Task Forces have
teamed up to create this Best Practice guide. Hopefully this guide,
together with the technical CMDI 1.2 specification, will be a valuable
knowledge base and will help any (technical) CMDI user to bring her CMD
records to their full potential use within
\href{https://www.clarin.eu}{CLARIN}.

\end{document}
