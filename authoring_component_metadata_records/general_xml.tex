\documentclass[]{article}
\usepackage{lmodern}
\usepackage{amssymb,amsmath}
\usepackage{ifxetex,ifluatex}
\usepackage{fixltx2e} % provides \textsubscript
\ifnum 0\ifxetex 1\fi\ifluatex 1\fi=0 % if pdftex
  \usepackage[T1]{fontenc}
  \usepackage[utf8]{inputenc}
\else % if luatex or xelatex
  \ifxetex
    \usepackage{mathspec}
  \else
    \usepackage{fontspec}
  \fi
  \defaultfontfeatures{Ligatures=TeX,Scale=MatchLowercase}
\fi
% use upquote if available, for straight quotes in verbatim environments
\IfFileExists{upquote.sty}{\usepackage{upquote}}{}
% use microtype if available
\IfFileExists{microtype.sty}{%
\usepackage{microtype}
\UseMicrotypeSet[protrusion]{basicmath} % disable protrusion for tt fonts
}{}
\usepackage{hyperref}
\hypersetup{unicode=true,
            pdfborder={0 0 0},
            breaklinks=true}
\urlstyle{same}  % don't use monospace font for urls
\usepackage{longtable,booktabs}
\IfFileExists{parskip.sty}{%
\usepackage{parskip}
}{% else
\setlength{\parindent}{0pt}
\setlength{\parskip}{6pt plus 2pt minus 1pt}
}
\setlength{\emergencystretch}{3em}  % prevent overfull lines
\providecommand{\tightlist}{%
  \setlength{\itemsep}{0pt}\setlength{\parskip}{0pt}}
\setcounter{secnumdepth}{0}
% Redefines (sub)paragraphs to behave more like sections
\ifx\paragraph\undefined\else
\let\oldparagraph\paragraph
\renewcommand{\paragraph}[1]{\oldparagraph{#1}\mbox{}}
\fi
\ifx\subparagraph\undefined\else
\let\oldsubparagraph\subparagraph
\renewcommand{\subparagraph}[1]{\oldsubparagraph{#1}\mbox{}}
\fi

\date{}

\begin{document}

\subsection{General XML}\label{general-xml}

X1: Include a reference to the profile XSD generated by the Component
Registry

{[}priority: high{]} {[}CLARIN B Centre requirement: 6.6{]} {[}check:
CMDI Instance Validator{]}

For each profile stored in the Component Registry a dynamically
generated XSD is available. The URL of this XSD is available in the Info
dialog of a profile and should be included in the schemaLocation
attribute on the CMD root element. This enables validation of a CMD
record by general XSD validators, including the CMDI Validator. The
Component Registry URL should be used as it ensures that fixes in the
transformation from a profile specification into an XSD are included in
the validation process.

X2: Use common namespace prefixes

{[}priority: low{]} {[}\emph{TODO: check: CMDI Instance Validator}{]}

Namespace prefixes are officially just syntactic sugar in XML, i.e.,
provide a convenient shortcut. However, using common prefixes enable
users to quickly assess the scope of an element. The CMDI 1.2
specification recommends the following prefixes for the namespaces URIs
in CMDI:

\begin{longtable}[c]{@{}llll@{}}
\toprule
\begin{minipage}[b]{0.05\columnwidth}\raggedright\strut
Prefix
\strut\end{minipage} &
\begin{minipage}[b]{0.05\columnwidth}\raggedright\strut
Namespace Name
\strut\end{minipage} &
\begin{minipage}[b]{0.05\columnwidth}\raggedright\strut
Comment
\strut\end{minipage} &
\begin{minipage}[b]{0.05\columnwidth}\raggedright\strut
Recommended Syntax
\strut\end{minipage}\tabularnewline
\midrule
\endhead
\begin{minipage}[t]{0.05\columnwidth}\raggedright\strut
cmd
\strut\end{minipage} &
\begin{minipage}[t]{0.05\columnwidth}\raggedright\strut
\url{http://www.clarin.eu/cmd/1}
\strut\end{minipage} &
\begin{minipage}[t]{0.05\columnwidth}\raggedright\strut
CMDI instance (general/envelope)
\strut\end{minipage} &
\begin{minipage}[t]{0.05\columnwidth}\raggedright\strut
prefixed
\strut\end{minipage}\tabularnewline
\begin{minipage}[t]{0.05\columnwidth}\raggedright\strut
cmdp
\strut\end{minipage} &
\begin{minipage}[t]{0.05\columnwidth}\raggedright\strut
\url{http://www.clarin.eu/cmd/1/profiles/\%7BprofileId\%7D}
\strut\end{minipage} &
\begin{minipage}[t]{0.05\columnwidth}\raggedright\strut
CMDI payload (profile specific)
\strut\end{minipage} &
\begin{minipage}[t]{0.05\columnwidth}\raggedright\strut
prefixed
\strut\end{minipage}\tabularnewline
\begin{minipage}[t]{0.05\columnwidth}\raggedright\strut
cue
\strut\end{minipage} &
\begin{minipage}[t]{0.05\columnwidth}\raggedright\strut
\url{http://www.clarin.eu/cmd/cues/1}
\strut\end{minipage} &
\begin{minipage}[t]{0.05\columnwidth}\raggedright\strut
Cues for tools
\strut\end{minipage} &
\begin{minipage}[t]{0.05\columnwidth}\raggedright\strut
prefixed
\strut\end{minipage}\tabularnewline
\begin{minipage}[t]{0.05\columnwidth}\raggedright\strut
xs
\strut\end{minipage} &
\begin{minipage}[t]{0.05\columnwidth}\raggedright\strut
\url{http://www.w3.org/2001/XMLSchema}
\strut\end{minipage} &
\begin{minipage}[t]{0.05\columnwidth}\raggedright\strut
XML Schema
\strut\end{minipage} &
\begin{minipage}[t]{0.05\columnwidth}\raggedright\strut
prefixed
\strut\end{minipage}\tabularnewline
\begin{minipage}[t]{0.05\columnwidth}\raggedright\strut
xsi
\strut\end{minipage} &
\begin{minipage}[t]{0.05\columnwidth}\raggedright\strut
\url{http://www.w3.org/2001/XMLSchema-instance}
\strut\end{minipage} &
\begin{minipage}[t]{0.05\columnwidth}\raggedright\strut
XML Schema instance
\strut\end{minipage} &
\begin{minipage}[t]{0.05\columnwidth}\raggedright\strut
prefixed
\strut\end{minipage}\tabularnewline
\bottomrule
\end{longtable}

See section \href{./Workflow.md}{3.4} regarding validation, which
implies well-formed XML.

X3: Use UTF-8 encoding

{[}priority: high{]}

\begin{longtable}[c]{@{}l@{}}
\toprule
Note by Menzo\tabularnewline
\midrule
\endhead
Could be split: specify encoding used (high), use UTF-8
(middle)\tabularnewline
\bottomrule
\end{longtable}

The encoding of a CMD record, i.e., XML documents in general, doesn't
have to be stated explicitly. It can be provided in various, possibly
conflicting ways: via a Byte Order Marker (BOM), in the XML declaration
of the document or a HTTP header. The best practice is to align all
these methods to express an UTF-8 encoding, but include at least the XML
declaration to indicate the encoding used.

\end{document}
